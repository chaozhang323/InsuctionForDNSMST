\documentclass[12pt, oneside]{article}
\usepackage{geometry}                		% See geometry.pdf to learn the layout options. There are lots.
\geometry{letterpaper} 
\usepackage{amsmath}
\usepackage{amsthm}
\usepackage{amssymb}
\usepackage{graphicx}
\usepackage{color}
\usepackage{float}
\usepackage{subfig}
\usepackage[flushleft]{threeparttable}
\usepackage{gensymb}
\usepackage{multirow}
\usepackage[dvipsnames]{xcolor}
\newcommand{\BibTeX}{\textsc{Bib}\TeX}
%opening
\title{Useful Commands}
\author{}

\begin{document}


\section{HDF5 Tools}

\subsection{HDFview.sh}
This tool can be used to view and edit HDF5 files. 

\noindent In order to use this tool use the command ``hdfview.sh'' followed by the file you wish to open.
Afterwards you are able to look at the data set and edit cells if needed.

\subsection{h5diff}
This tool is used in order to compare HDF5 files and output difference in them. The command for this tool is:

\begin{verbatim}
 h5diff [OPTIONS] file1 file2 object1 object2
\end{verbatim}

\noindent The only required parts are the 2 files to be compared. If no options are selected the tool will report how many differences there are and
where they were found, if no differences are found there will be no output. Objects can be selected for comparison these can be groups or datasets, for example /T.

\noindent Some useful options are: 

-r which will report the difference of each value compared or will report no difference if there are no differences

-d delta(this should be a number)  which will only report differences greater than delta
\newline
An example use of this tool would be comparing the results of 2 different DNS runs to validate a new code version. This would be done by using the command
``h5diff -r run1 run2 /T '' in order to check for differences in the temperature calculated. The -r option will make it output 

\begin{verbatim}
dataset: </T> and </T>
0 differences found
\end{verbatim}
if there are no differences. Without -r there would be no output 


\subsection{h5ls}

This tool is used to output information about a HDF5 file, it will report the name and size of each data set of a file.
The command for this tool is:

\begin{verbatim}
 h5diff [OPTIONS] file1/object file2/object ... 
\end{verbatim}

\noindent The /object part can be neglected and will result in information about all datasets of a file being outputted, if included only the dataset referenced by object will be displayed

\newline
An example use of this tool is finding out the size of the T data set in an HDF5 file contains. This could be done using the command ``h5ls flowdata\_i0001\_0123\_00250000.h5/T'' which outputs

\begin{verbatim}
 T                        Dataset {1, 123, 295}
\end{verbatim}
\newline
This shows that the dataset T exists and has the dimensions [1,123,295]



\subsection{h5dump}

This command is used to examine and output an HDF5 file into a readable format.

\noindent The command for this tool is

\begin{verbatim}
 h5dump [options] file
\end{verbatim}

\noindent Some useful options are

 -d [object]   this will select a specific dataset, object is a data item like u.

 -p  This will add properties of a file to the outputs. This can be useful when looking at VDS files as it will allow you to inspect the links that make up the file.
It should be noted that in order to inspect VDS files a tool version of at least HDF5\_1.10 or later will be needed.

\newline
An example use of this command is finding out what the source files of a VDS are. This could be done with the command 
\begin{verbatim}
 h5dump -p VDS_flowdata_00250000.h5  > props
\end{verbatim}

This will output the data from the file VDS\_flowdata\_00250000.h5 to a file named props. This will display source files like

\begin{verbatim}
SOURCE {
  FILE "flowdata_i0124_0246_00250000.h5"
  DATASET "u"
  SELECTION ALL
}
\end{verbatim}

\section{Module}

Module is a command used for making use of and finding modules on a computer.

\noindent Some uses of the command are:

Loading modules: this is done with the command    ``module load [module name]''.  For example if you wanted to use HDF5 version 10 on bluewaters, you would use the command
   ``load module cray-hdf5/1.10.0''
\newline
   
Displaying available modules: this is done with the command    ``module avail [module name]''.  This command will display all available modules or if [module name] is filled in, 
available modules that include the name
\newline

Finding paths to modules: this is done with the command    ``module show [module name]''.    This command will display the location of the module name you give it. This is helpful for finding 
the location to use for compiling code.








\end{document}